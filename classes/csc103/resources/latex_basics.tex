\documentclass[]{article}

\title{Example of Basics In LaTeX}
\author{Pablo Frank Bolton}
\date{August 2021}

\begin{document}

\maketitle

\section{This is a Top Section}

This is normal text with \textbf{bold} and \textit{italics}, both of which have shortcuts.

Note that a single newline in the .tex file is ignored. 
To have a new line in the pdf you need at least two newlines in the .tex file. This will cause the next line to be indented (new paragraph)


This is an actual new line. I'll add a gap between this line and the next.

\medskip

This is a new paragraph with a gap before it.


\subsection{This is a subsection}

The following is a numbered list:
\begin{enumerate}
    \item you could format the numbering differently, but that is for later
    \item this is another item
\end{enumerate}

Now let's see an itemized list with bullet points:

\begin{itemize}
    \item no numbers here, just bullets
    \item these bullets can be formatted too
\end{itemize}


\section{Math environment and symbols}

There are two types of math environments:

\subsection{Inline}
The inline environment, lets you write things like: \( \Lambda_{i} = x^{2\pi-\omega} + y^{j}_{i} - \frac{1}{\delta_i + 1} \)

Another way to do it is like this: $ y = mx + b$

you can also write it like this:
\begin{math}
    E = MC^2
\end{math}

You can add text inside an equation by using the \texttt{``\textbackslash textrm\{ \}''} 
environment like so:
\begin{math}
        \mu = \textrm{average height} 
\end{math}

\subsection{Display}

You can do "Display" math with:
\[ \Lambda_{i} = x^{2\pi-\omega} + y^{j}_{i} - \frac{1}{\delta_i + 1} \]

or this:

\begin{equation}
    E = MC^2
\end{equation}

or this:

\begin{displaymath}
    h^2 = a^2 + b^2
\end{displaymath}

We'll explain how to do fancier formatting at a later date.


\end{document}